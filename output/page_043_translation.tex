\documentclass[a4paper,12pt]{article}
\usepackage{xltxtra}
\usepackage{xeCJK}
\usepackage{geometry}
\usepackage{xcolor}
\usepackage{titlesec}
\usepackage{parskip}

\geometry{left=2cm,right=2cm,top=2cm,bottom=2cm}

% Fonts
\setmainfont{Noto Serif}
\setCJKmainfont{Noto Serif CJK SC}
\newCJKfontfamily\zhfont{Noto Serif CJK SC}
\newCJKfontfamily\jafont{Noto Serif CJK JP}

% Colors
\definecolor{rucolor}{RGB}{0,0,0}      % Black
\definecolor{encolor}{RGB}{0,0,139}    % DarkBlue
\definecolor{zhcolor}{RGB}{139,0,0}    % DarkRed
\definecolor{jacolor}{RGB}{0,100,0}    % DarkGreen

% Translation Macro
\newcommand{\parallelsentence}[4]{
    \noindent
    \textbf{\textcolor{rucolor}{RU:}} #1 \\
    \textbf{\textcolor{encolor}{EN:}} #2 \\
    \textbf{\textcolor{zhcolor}{ZH:}} {\zhfont #3} \\
    \textbf{\textcolor{jacolor}{JA:}} {\jafont #4}
    \vspace{0.5cm}
}

\begin{document}

\section*{Page 43 Translation}
\parallelsentence
{Facebook подсказал не как надо делать, а как не надо, -- объяснял Дуров. -- Одноклассники -- тоже. Важно было понять, от чего избавиться.}
{Facebook showed us not how to do things, but how not to, Durov explained. Odnoklassniki too. The key was understanding what to eliminate.}
{Facebook告诉我们的不是该怎么做,而是不该怎么做,杜罗夫解释道。Odnoklassniki(同学网)也一样。关键是弄清楚该放弃什么。}
{Facebookは、どうすべきかではなく、どうすべきでないかを教えてくれた、とドゥーロフは説明した。Odnoklassnikiも同様だ。何を排除すべきかを理解することが重要だった。}

\parallelsentence
{Я понял, что главное -- стартовая страница пользователя. Человек хочет видеть свой профиль, а не ленту происходящего у друзей и предложение разных возможностей, как у Facebook.}
{I realized that the key was the user's landing page. People want to see their own profile, not a feed of what's happening with friends and suggestions of various features like on Facebook.}
{我意识到关键在于用户的首页。人们想看到的是自己的个人资料页面,而不是像Facebook那样显示朋友动态和各种功能推荐。}
{ユーザーのランディングページが鍵だと気づいた。人々は友達の動向フィードやFacebookのような機能提案ではなく、自分のプロフィールを見たいのだ。}

\parallelsentence
{Надо выводить личную страницу как стартовую, чтобы человек загружал больше личных данных, чтобы лепил идеального себя.}
{You need to make the personal page the landing page so that users upload more personal data, so they craft their ideal selves.}
{要把个人页面设为首页,这样用户就会上传更多个人数据,塑造理想中的自己。}
{個人ページをランディングページにして、ユーザーがより多くの個人データをアップロードし、理想の自分を作り上げられるようにする必要がある。}

\parallelsentence
{Догадка Дурова не была оригинальной -- с момента возникновения социальных сервисов стало очевидно, что у человека возникают отношения со своей страницей, профилем.}
{Durov's insight was not entirely original -- since the emergence of social services, it had become obvious that people develop relationships with their pages, their profiles.}
{杜罗夫的想法并非完全原创——自从社交服务出现以来,人们与自己的页面、个人资料之间产生情感联系这一点已经很明显了。}
{ドゥーロフの洞察は完全にオリジナルではなかった——ソーシャルサービスの登場以来、人々が自分のページやプロフィールと関係を築くことは明らかになっていた。}

\parallelsentence
{Люди не только загружают новые фотографии, но и периодически редактируют данные, любимые цитаты, предпочтения, создают свой образ.}
{People do not just upload new photos -- they periodically edit their data, favorite quotes, preferences, creating their own image.}
{人们不仅上传新照片,还会定期编辑自己的资料、喜欢的名言、偏好设置,塑造自己的形象。}
{人々は新しい写真をアップロードするだけでなく、定期的にデータ、好きな引用、好みを編集し、自分のイメージを作り上げる。}

\parallelsentence
{Но ставить как стартовую страницу сам профиль и стену -- это как раз чисто русская находка.}
{But making the profile and wall the landing page -- that was a purely Russian innovation.}
{但把个人资料页和动态墙作为首页——这正是纯粹的俄罗斯式创新。}
{しかし、プロフィールとウォールをランディングページにすること——それは純粋にロシア的な革新だった。}

\end{document}