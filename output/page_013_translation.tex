\documentclass[a4paper,12pt]{article}
\usepackage{xltxtra}
\usepackage{xeCJK}
\usepackage{geometry}
\usepackage{xcolor}
\usepackage{titlesec}
\usepackage{parskip}

\geometry{left=2cm,right=2cm,top=2cm,bottom=2cm}

% Fonts
\setmainfont{Noto Serif}
\setCJKmainfont{Noto Serif CJK SC}
\newCJKfontfamily\zhfont{Noto Serif CJK SC}
\newCJKfontfamily\jafont{Noto Serif CJK JP}

% Colors
\definecolor{rucolor}{RGB}{0,0,0}      % Black
\definecolor{encolor}{RGB}{0,0,139}    % DarkBlue
\definecolor{zhcolor}{RGB}{139,0,0}    % DarkRed
\definecolor{jacolor}{RGB}{0,100,0}    % DarkGreen

% Translation Macro
\newcommand{\parallelsentence}[4]{
    \noindent
    \textbf{\textcolor{rucolor}{RU:}} #1 \\
    \textbf{\textcolor{encolor}{EN:}} #2 \\
    \textbf{\textcolor{zhcolor}{ZH:}} {\zhfont #3} \\
    \textbf{\textcolor{jacolor}{JA:}} {\jafont #4}
    \vspace{0.5cm}
}

\begin{document}

\section*{Page 13 Translation}
\parallelsentence
{Мальчик с томом Сервантеса выходит из подъезда, огибает автомобиль, который какой-то негодяй поставил так, что пешеходы еле протискиваются мимо, и сворачивает за угол.}
{A boy with a volume of Cervantes exits the building entrance, skirts around a car that some scoundrel parked so badly that pedestrians can barely squeeze past, and turns the corner.}
{一个手捧塞万提斯著作的男孩走出公寓楼门口,绕过一辆被某个混蛋停得让行人几乎无法通过的汽车,然后拐过街角。}
{セルバンテスの本を抱えた少年が建物の入り口から出てきて、誰かの非常識な駐車で歩行者がかろうじて通れるような車を迂回し、角を曲がる。}

\parallelsentence
{Перед ним пустынные кварталы, поля и высоковольтные вышки, а в физиономию дует ветер – как везде в Петербурге, но в этом районе особенно. Рядом море.}
{Before him stretch deserted blocks, fields, and high-voltage towers, while the wind blows in his face—as it does everywhere in St. Petersburg, but especially in this district. The sea is nearby.}
{他面前是荒凉的街区、田野和高压输电塔,风吹在他脸上——这在彼得堡随处可见,但这个区域尤其明显。附近就是大海。}
{目の前には荒涼とした街区、野原、高圧鉄塔が広がり、風が彼の顔に吹き付ける――サンクトペテルブルクではどこでもそうだが、この地区では特に強い。海がすぐそばにある。}

\parallelsentence
{Архитектор раскрасил панели домов в оранжевый и бордовый, чтобы однообразное серое не свело район с ума.}
{The architect painted the building panels orange and burgundy so that the monotonous gray wouldn't drive the district mad.}
{建筑师把楼房的外墙板涂成橙色和酒红色,以免单调的灰色让这个区域的人发疯。}
{建築家は建物のパネルをオレンジ色とバーガンディに塗り、単調なグレーが地区を狂わせないようにした。}

\parallelsentence
{Расстояния между корпусами напоминают о заполярных городах, где возводить что-либо можно лишь на сопках, а дворы имеют сторону в километр.}
{The distances between buildings recall polar cities where construction is only possible on hills, and courtyards stretch a kilometer on each side.}
{建筑之间的距离让人想起北极圈城市,那里只能在山丘上盖楼,而庭院的边长可达一公里。}
{建物間の距離は極北の都市を思わせる。そこでは丘の上にしか建設できず、中庭の一辺は1キロメートルにも及ぶ。}

\parallelsentence
{Летом они зарастают одуванчиками, разнотравьем и камышом. Вокруг осушенные болота.}
{In summer they become overgrown with dandelions, wildflowers, and reeds. Drained marshes surround the area.}
{夏天,这里长满了蒲公英、各种野草和芦苇。周围是排干的沼泽地。}
{夏になると、タンポポ、野草、葦が生い茂る。周囲には干拓された湿地帯が広がる。}

\end{document}