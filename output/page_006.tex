\documentclass[11pt,a4paper]{article}
\usepackage{xeCJK}
\usepackage{fontspec}
\usepackage{xcolor}
\usepackage{geometry}
\usepackage{setspace}
\usepackage{parskip}

\geometry{margin=2.5cm}
\setstretch{1.2}

\setmainfont{DejaVu Serif}
\setCJKmainfont{Noto Sans CJK SC}
\setCJKfamilyfont{zhsong}{Noto Serif CJK SC}
\setCJKfamilyfont{jpgothic}{Noto Sans CJK JP}

\definecolor{russian}{RGB}{0,0,0}
\definecolor{english}{RGB}{0,102,204}
\definecolor{chinese}{RGB}{204,0,0}
\definecolor{japanese}{RGB}{0,153,76}

\newcommand{\ru}[1]{{\color{russian}#1}}
\newcommand{\en}[1]{{\color{english}#1}}
\newcommand{\zh}[1]{{\color{chinese}\CJKfamily{zhsong}#1}}
\newcommand{\ja}[1]{{\color{japanese}\CJKfamily{jpgothic}#1}}

\begin{document}

\begin{center}
{\Large\textbf{Page 6 - Preface (conclusion)}}\\
\vspace{0.2cm}
{\small Код Дурова | Yuri Saprykin}
\end{center}

\vspace{0.5cm}

\ru{Возможно, правильнее было бы подойти к этому тексту как к авантюрному роману, серии удивительных приключений и сказочных перевоплощений – развязка которых, кажется, далеко не очевидна.}

\en{Perhaps it would be more correct to approach this text as an adventure novel, a series of amazing adventures and fairy-tale transformations—whose denouement seems far from obvious.}

\zh{也许更正确的做法是将这部作品视为一部冒险小说,一系列令人惊叹的冒险和童话般的变形——其结局似乎远非显而易见。}

\ja{おそらく、このテキストを冒険小説、驚くべき冒険とおとぎ話のような変身の連続として扱う方が正しいだろう。その結末はまったく明白ではないようだ。}

\vspace{0.4cm}

\ru{Герой этой книги не борется за свободу – а утверждает ее самим фактом своего существования.}

\en{The hero of this book does not fight for freedom—but affirms it by the very fact of his existence.}

\zh{这本书的主人公不是为自由而战——而是通过他存在的事实本身来确认自由。}

\ja{この本の主人公は自由のために戦うのではなく、彼の存在そのものによって自由を主張している。}

\vspace{0.4cm}

\ru{В любом случае, это открывает и перед автором, и перед читателями блистательную перспективу: в том смысле, что продолжение следует.}

\en{In any case, this opens a brilliant prospect for both the author and readers: in the sense that the story continues.}

\zh{无论如何,这为作者和读者都打开了一个辉煌的前景:从这个意义上说,故事还在继续。}

\ja{いずれにせよ、これは著者と読者の両方に輝かしい展望を開く。つまり、物語は続くということだ。}

\vspace{0.6cm}

\ru{Юрий Сапрыкин, публицист, шеф-редактор «Рамблер-Афиша»}

\en{Yuri Saprykin, publicist, editor-in-chief of Rambler-Afisha}

\zh{尤里·萨普雷金,评论家,Rambler-Afisha主编}

\ja{ユーリ・サプリキン、評論家、ランブラー・アフィーシャ編集長}

\end{document}
