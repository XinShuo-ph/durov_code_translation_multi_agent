\documentclass[11pt,a4paper]{article}
\usepackage{xeCJK}
\usepackage{fontspec}
\usepackage{xcolor}
\usepackage{geometry}
\usepackage{setspace}
\usepackage{parskip}

% Page setup
\geometry{margin=2.5cm}
\setstretch{1.2}

% Font setup
\setmainfont{DejaVu Serif}
\setCJKmainfont{Noto Sans CJK SC}
\setCJKfamilyfont{zhsong}{Noto Serif CJK SC}
\setCJKfamilyfont{jpgothic}{Noto Sans CJK JP}

% Colors for each language
\definecolor{russian}{RGB}{0,0,0}
\definecolor{english}{RGB}{0,102,204}
\definecolor{chinese}{RGB}{204,0,0}
\definecolor{japanese}{RGB}{0,153,76}

% Commands
\newcommand{\ru}[1]{{\color{russian}#1}}
\newcommand{\en}[1]{{\color{english}#1}}
\newcommand{\zh}[1]{{\color{chinese}\CJKfamily{zhsong}#1}}
\newcommand{\ja}[1]{{\color{japanese}\CJKfamily{jpgothic}#1}}

\begin{document}

\begin{center}
{\Large\textbf{Page 13 - Chapter 1: Botanical Garden}}\\
\vspace{0.3cm}
{\small Demo Translation | Код Дурова}
\end{center}

\vspace{0.5cm}

\ru{Мальчик с томом Сервантеса выходит из подъезда, огибает автомобиль, который какой-то негодяй поставил так, что пешеходы еле протискиваются мимо, и сворачивает за угол.}

\en{A boy with a volume of Cervantes exits the apartment entrance, walks around a car that some scoundrel parked so pedestrians can barely squeeze past, and turns the corner.}

\zh{一个手捧塞万提斯著作的男孩走出公寓楼,绕过一辆被某个混蛋停在那里让行人几乎无法通过的汽车,然后转过街角。}

\ja{セルバンテスの本を持った少年が建物の入り口を出て、誰かが歩行者がやっと通れるように停めた車を避けて、角を曲がる。}

\vspace{0.4cm}

\ru{Перед ним пустынные кварталы, поля и высоковольтные вышки, а в физиономию дует ветер – как везде в Петербурге, но в этом районе особенно.}

\en{Before him lie desolate blocks, fields, and high-voltage towers, while wind blows in his face—as everywhere in Petersburg, but especially fierce in this district.}

\zh{眼前是荒凉的街区、空地和高压电塔,风吹在脸上——圣彼得堡到处都有风,但这个区域尤其强烈。}

\ja{目の前には荒涼とした街区、空き地、高圧送電塔が広がり、顔に風が吹きつける。ペテルブルクではどこでもそうだが、この地区では特に強い。}

\vspace{0.4cm}

\ru{Рядом море.}

\en{The sea is nearby.}

\zh{大海就在附近。}

\ja{海が近くにある。}

\vspace{0.4cm}

\ru{Сканируя пространство на предмет гопников, мальчик с книгой идет к трассе.}

\en{Scanning the area for gopniki, the boy with the book walks toward the highway.}

\zh{男孩扫视周围是否有街头混混(гопники),然后朝公路走去。}

\ja{少年は本を持ちながら、ゴプニキ(街のチンピラ)がいないか辺りを見回しながら幹線道路へ向かう。}

\vspace{0.6cm}

{\footnotesize
\textbf{Translator's Notes:}
\begin{itemize}
\item \textit{Gopniki} (гопники): Russian street youth subculture; no direct English equivalent
\item \textit{Cervantes}: Author of Don Quixote, Pavel Durov's favorite book
\item Context: Describes young Pavel's childhood in peripheral St. Petersburg
\end{itemize}
}

\end{document}
