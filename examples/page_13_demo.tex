\documentclass[a4paper,10pt]{article}
\usepackage{geometry}
\geometry{top=2cm, bottom=2cm, left=2cm, right=2cm}
\usepackage{fontspec}
\usepackage{xeCJK}
\usepackage{paracol} 

% Fonts
\setmainfont{DejaVu Serif}
\newfontfamily\cyrillicfont{DejaVu Serif}
\newfontfamily\englishfont{DejaVu Serif}
\setCJKmainfont{AR PL UMing CN}
\newCJKfontfamily\chinesefont{AR PL UMing CN}
\newCJKfontfamily\japanesefont{IPAexMincho}

\title{Durov Code - Page 13 Demo}
\date{}

\begin{document}

\section*{Page 13}

\begin{paracol}{4}
\textbf{Russian} \switchcolumn
\textbf{English} \switchcolumn
\textbf{Chinese} \switchcolumn
\textbf{Japanese}
\end{paracol}

\vspace{0.5cm}

\begin{paracol}{4}
\textbf{Глава 1. Ботанический сад} \switchcolumn
\textbf{Chapter 1. Botanical Garden} \switchcolumn
{\chinesefont \textbf{第一章 植物园}} \switchcolumn
{\japanesefont \textbf{第1章 植物園}}
\end{paracol}

\vspace{0.5cm}

\begin{paracol}{4}
Мальчик с томом Сервантеса выходит из подъезда, огибает автомобиль, который какой-то негодяй поставил так, что пешеходы еле протискиваются мимо, и сворачивает за угол. \switchcolumn
A boy with a volume of Cervantes steps out of the entrance, skirts a car that some scoundrel parked so that pedestrians barely squeeze by, and turns the corner. \switchcolumn
{\chinesefont 一个手捧塞万提斯著作的男孩走出公寓楼,绕过一辆不知哪个混蛋乱停的车——那车停得让人只能勉强挤过去——然后拐过街角。} \switchcolumn
{\japanesefont セルバンテスの本を持った少年が建物の入り口を出て、誰か不届き者が停めた車の脇を通り抜ける。歩行者がやっと通れるほどの隙間しかないその車を避け、彼は角を曲がった。}
\end{paracol}

\vspace{0.3cm}

\begin{paracol}{4}
Перед ним пустынные кварталы, поля и высоковольтные вышки, а в физиономию дует ветер – как везде в Петербурге, но в этом районе особенно. \switchcolumn
Before him lie deserted blocks, fields, and high-voltage pylons, and the wind blows in his face—like everywhere in St. Petersburg, but especially in this district. \switchcolumn
{\chinesefont 眼前是空荡荡的街区、田野和高压线塔,风直扑面门——在圣彼得堡到处都是这样,但这片区域尤甚。} \switchcolumn
{\japanesefont 目の前には人けのない街区、野原、そして高圧送電線が広がり、風が顔に吹き付ける――サンクトペテルブルクではどこでもそうだが、この地区では特に強い。}
\end{paracol}

\vspace{0.3cm}

\begin{paracol}{4}
Рядом море. \switchcolumn
The sea is nearby. \switchcolumn
{\chinesefont 大海就在附近。} \switchcolumn
{\japanesefont すぐそばに海がある。}
\end{paracol}

\vspace{0.3cm}

\begin{paracol}{4}
Архитектор раскрасил панели домов в оранжевый и бордовый, чтобы однообразное серое не свело район с ума. \switchcolumn
The architect painted the house panels orange and burgundy so the monotonous gray wouldn't drive the district crazy. \switchcolumn
{\chinesefont 建筑师把房屋面板涂成了橙色和酒红色,以免单调的灰色把整个街区逼疯。} \switchcolumn
{\japanesefont 建築家は家のパネルをオレンジとワインレッドに塗った。単調な灰色がこの地区を狂わせないようにするためだ。}
\end{paracol}

\end{document}
