\documentclass[12pt,a4paper]{article}
\usepackage{xeCJK}
\usepackage{fontspec}
\usepackage{color}
\usepackage{geometry}
\usepackage{setspace}
\usepackage{enumitem}

% Set up fonts
\setmainfont{Noto Serif}
\setCJKmainfont{Noto Sans CJK SC}
\setCJKmonofont{Noto Sans Mono CJK SC}

% Define colors for each language
\definecolor{russiancolor}{RGB}{0,0,0}      % Black
\definecolor{englishcolor}{RGB}{0,68,153}   % Blue
\definecolor{chinesecolor}{RGB}{204,0,0}    % Red
\definecolor{japanesecolor}{RGB}{0,128,0}   % Green

% Page geometry
\geometry{
    a4paper,
    left=2.5cm,
    right=2.5cm,
    top=3cm,
    bottom=3cm
}

% Line spacing
\setstretch{1.3}

% Commands for each language
\newcommand{\ru}[1]{\textcolor{russiancolor}{#1}}
\newcommand{\en}[1]{\textcolor{englishcolor}{#1}}
\newcommand{\zh}[1]{\textcolor{chinesecolor}{#1}}
\newcommand{\ja}[1]{\textcolor{japanesecolor}{#1}}

% Sentence block environment
\newenvironment{sentenceblock}{%
    \begin{spacing}{1.5}
    \setlength{\parindent}{0pt}
    \setlength{\parskip}{0.8em}
}{%
    \end{spacing}
}

\begin{document}

% Title page
\begin{titlepage}
    \centering
    \vspace*{3cm}
    
    {\Huge\bfseries Код Дурова}\\[0.5cm]
    {\LARGE Durov Code}\\[0.3cm]
    {\LARGE 杜罗夫密码}\\[0.3cm]
    {\LARGE ドゥーロフ・コード}\\[2cm]
    
    {\Large Multilingual Edition}\\[0.5cm]
    {\large Russian · English · 中文 · 日本語}\\[3cm]
    
    {\large Николай В. Кononov}\\[0.3cm]
    {\large Nikolai V. Kononov}\\[5cm]
    
    {\large 2026}
\end{titlepage}

\newpage

% Format explanation
\section*{About This Edition}

This multilingual edition presents the complete Russian text of ``Код Дурова'' (Durov Code) alongside English, Chinese (中文), and Japanese (日本語) translations.

\textbf{Color coding:}
\begin{itemize}[leftmargin=2cm]
    \item[\ru{■}] Russian (Русский) - Black
    \item[\en{■}] English - Blue
    \item[\zh{■}] Chinese (中文) - Red
    \item[\ja{■}] Japanese (日本語) - Green
\end{itemize}

Each sentence appears in all four languages in sequence, allowing readers to compare translations and appreciate linguistic nuances.

\newpage

% Demo page content (Page 13 excerpt as example)
\section*{Page 13 - Chapter 1: Botanical Garden}
\thispagestyle{empty}

\begin{sentenceblock}

\ru{Мальчик с томом Сервантеса выходит из подъезда, огибает автомобиль, который какой-то негодяй поставил так, что пешеходы еле протискиваются мимо, и сворачивает за угол.}

\en{A boy carrying a volume of Cervantes exits the building entrance, circles around a car that some scoundrel parked so tightly that pedestrians can barely squeeze past, and turns the corner.}

\zh{一个手捧塞万提斯著作的男孩走出公寓楼入口,绕过一辆被某个混蛋停得让行人几乎挤不过去的汽车,然后转过街角。}

\ja{セルバンテスの本を持った少年が建物の入り口を出て、誰かの不届き者が歩行者がやっと通れるほど狭く駐車した車を回り込み、角を曲がる。}

\bigskip

\ru{Перед ним пустынные кварталы, поля и высоковольтные вышки, а в физиономию дует ветер – как везде в Петербурге, но в этом районе особенно.}

\en{Before him lie deserted blocks, fields and high-voltage towers, and wind blows in his face - as it does everywhere in Petersburg, but especially in this district.}

\zh{他面前是荒凉的街区、田野和高压电塔,风吹在脸上——圣彼得堡到处都是这样,但这个区域尤其如此。}

\ja{彼の前には人気のない街区、草原、高圧送電塔が広がり、顔に風が吹き付ける——ペテルブルクではどこでもそうだが、この地区では特にひどい。}

\bigskip

\ru{Рядом море.}

\en{The sea is nearby.}

\zh{大海就在附近。}

\ja{海は近い。}

\end{sentenceblock}

\vfill

\begin{center}
\small
\textit{Page 13 continues...}
\end{center}

\end{document}
