% Page 13 demo translation (RU/EN/ZH/JA) — quick pass
\documentclass[11pt]{article}

\usepackage[a4paper,margin=1.5cm]{geometry}
\usepackage{fontspec}
\usepackage{xeCJK}
\usepackage{xcolor}
\usepackage{setspace}
\usepackage{enumitem}

\setmainfont{Noto Serif}
\setsansfont{Noto Sans}
\setCJKmainfont{Noto Serif CJK SC}
\newCJKfontfamily\CJKjp{Noto Serif CJK JP}

\definecolor{enblue}{HTML}{1E5AA8}
\definecolor{zhred}{HTML}{B01E1E}
\definecolor{jagreen}{HTML}{167A3A}

\newcommand{\ru}[1]{\textcolor{black}{#1}}
\newcommand{\en}[1]{\textcolor{enblue}{#1}}
\newcommand{\zh}[1]{\textcolor{zhred}{#1}}
\newcommand{\ja}[1]{{\CJKjp\textcolor{jagreen}{#1}}}

\setlength{\parindent}{0pt}
\setlist[itemize]{leftmargin=*,itemsep=0.55em}
\linespread{1.03}

\begin{document}

{\large \textbf{Page 13 — Demo translation}}\par
\vspace{0.6em}

\begin{itemize}
  \item \ru{Глава 1}
  
        \en{Chapter 1}
  
        \zh{第一章}
  
        \ja{第1章}

  \item \ru{Ботанический сад}
  
        \en{Botanical Garden}
  
        \zh{植物园}
  
        \ja{植物園}

  \item \ru{Мальчик с томом Сервантеса выходит из подъезда, огибает автомобиль, который какой-то негодяй поставил так, что пешеходы еле протискиваются мимо, и сворачивает за угол.}
  
        \en{A boy with a volume of Cervantes leaves the entranceway, skirts a car some jerk has parked so tight pedestrians can barely squeeze past, and turns the corner.}
  
        \zh{一个抱着塞万提斯作品集的男孩走出楼道,绕过一辆被某个混蛋停得让行人几乎挤不过去的车,拐进了街角。}
  
        \ja{セルバンテスの本を持った少年が建物の入口を出て、誰かのろくでなしが歩行者がやっとすり抜けられるように停めた車を回り込み、角を曲がる。}

  \item \ru{Перед ним пустынные кварталы, поля и высоковольтные вышки, а в физиономию дует ветер – как везде в Петербурге, но в этом районе особенно.}
  
        \en{Ahead of him are deserted blocks, fields, and high-voltage pylons, and the wind blows straight into his face—like everywhere in Petersburg, but especially in this district.}
  
        \zh{他面前是空荡荡的街区、田野和高压电塔,风直往脸上吹——在彼得堡到处都有风,但在这片地方尤其凶。}
  
        \ja{目の前には人気のない街区、原っぱ、高圧線の鉄塔が広がり、風が顔面に叩きつける――ペテルブルクではどこでも風が吹くが、この地区はとりわけだ。}

  \item \ru{Рядом море.}
  
        \en{The sea is nearby.}
  
        \zh{海就在旁边。}
  
        \ja{海が近い。}

  \item \ru{Архитектор раскрасил панели домов в оранжевый и бордовый, чтобы однообразное серое не свело район с ума.}
  
        \en{An architect painted the prefab panels orange and burgundy so the monotonous gray wouldn’t drive the neighborhood insane.}
  
        \zh{建筑师把楼房的预制板刷成橙色和酒红色,好让那种单调的灰不至于把整个街区逼疯。}
  
        \ja{単調な灰色がこの地区を狂わせないようにと、建築家が団地のパネルをオレンジとえんじ色に塗った。}

  \item \ru{Расстояния между корпусами напоминают о заполярных городах, где возводить что-либо можно лишь на сопках, а дворы имеют сторону в километр.}
  
        \en{The spacing between the buildings brings to mind Arctic towns, where you can build only on hillsides, and courtyards run a kilometer across.}
  
        \zh{楼与楼之间的距离让人想起北极圈附近的城市:那里只能在丘岗上盖东西,院子的一边能有一公里长。}
  
        \ja{棟と棟の間隔は北極圏の都市を思わせる。そこでは丘の斜面にしか建てられず、団地の中庭は一辺が一キロにもなる。}

  \item \ru{Летом они зарастают одуванчиками, разнотравьем и камышом.}
  
        \en{In summer they get overgrown with dandelions, wild grasses, and reeds.}
  
        \zh{夏天里,那里长满蒲公英、杂草和芦苇。}
  
        \ja{夏になると、タンポポや雑草、ヨシが一面を覆う。}

  \item \ru{Вокруг осушенные болота.}
  
        \en{All around are drained swamps.}
  
        \zh{四周是被排干的沼泽地。}
  
        \ja{周囲は干拓された湿地だ。}

  \item \ru{Сканируя пространство на предмет гопников, мальчик с книгой идет к трассе.}
  
        \en{Scanning the area for gopniks, the boy with the book heads toward the highway.}
  
        \zh{男孩抱着书,一边警惕地扫视四周有没有“街头混混”,一边朝公路走去。}
  
        \ja{ゴプニクがいないか周囲を探りながら、本を持った少年は幹線道路へ向かう。}

  \item \ru{Некоторые многоэтажки недостроены, а улицы недочерчены.}
  
        \en{Some high-rises are unfinished, and some streets were never fully laid out.}
  
        \zh{有些高楼烂尾了,街道也只画到一半。}
  
        \ja{未完成の高層住宅もあれば、道路も描きかけのまま終わっている。}

  \item \ru{У одного корпуса поставили стилобат, а потом, видимо, куда-то просадили деньги – осталась бетонная коробка.}
  
        \en{For one building they poured the podium slab, and then—apparently—the money vanished somewhere, leaving a concrete shell.}
  
        \zh{有一栋楼先做了基座平台(台基),后来钱显然不知被“挪”到哪去了,只剩下一个混凝土盒子。}
  
        \ja{ある棟では基壇だけ作って、その後は――おそらく金がどこかへ消えたのだろう――コンクリートの箱が残った。}

  \item \ru{Внутри нее горят костры и сидят парни, курят, треплются и малюют граффити.}
  
        \en{Inside, campfires burn and guys sit around smoking, jawing, and scrawling graffiti.}
  
        \zh{里面点着篝火,几个小子坐着抽烟、闲扯、涂鸦。}
  
        \ja{中では焚き火が燃え、若い連中が座り込んで煙草を吸い、だべり、落書きをしている。}

  \item \ru{Весной разливаются лужи, и парни сколачивают плоты, чтобы перебраться с континента «Камышовая» на континент «Ситцевая».}
  
        \en{In spring the puddles spread out, and the guys lash together rafts to get from the “Kamyshovaya” continent to the “Sittsevaya” continent.}
  
        \zh{春天积水漫开,那些小子就钉木筏,好从“芦苇街大陆”漂到“印花布街大陆”。}
  
        \ja{春になると水たまりがあふれ、連中は「カミショーヴァヤ大陸」から「シッツェヴァヤ大陸」へ渡るために筏を組む。}

  \item \ru{Маячат котлованы, наполненные мутной водой.}
  
        \en{Excavation pits loom, filled with muddy water.}
  
        \zh{远处隐约可见灌满浑水的基坑。}
  
        \ja{濁った水で満たされた掘り坑がちらつく。}

  \item \ru{За ними чернеет лес.}
  
        \en{Beyond them the forest darkens.}
  
        \zh{再往后,是一片发黑的树林。}
  
        \ja{その向こうに森が黒々と見える。}

  \item \ru{Перепрыгивая через лужи, мальчик проходит недострой и выбирается к дороге, по которой век назад возили торф.}
  
        \en{Hopping over puddles, the boy passes the unfinished building and makes his way to a road that, a century ago, carried peat.}
  
        \zh{男孩跳过一滩滩水,穿过烂尾楼,走到一条一百年前曾用来运泥炭的路上。}
  
        \ja{水たまりを飛び越えながら、少年は未完の建物を抜け、百年前には泥炭を運んでいたという道へ出る。}

  \item \ru{Справа забор кладбища и березки у надгробий.}
  
        \en{On the right is the cemetery fence and little birches by the gravestones.}
  
        \zh{右边是墓地的围栏,墓碑旁长着几棵小白桦。}
  
        \ja{右手には墓地の柵と、墓標のそばの白樺の若木。}

  \item \ru{Слева перекопанная площадь, окруженная скелетами панельных многоквартирных гигантов.}
  
        \en{On the left is a torn-up square ringed by the skeletons of giant prefab apartment blocks.}
  
        \zh{左边是一片被翻得乱七八糟的广场,四周围着巨型板楼的骨架。}
  
        \ja{左手には掘り返された広場があり、その周りを巨大なパネル式集合住宅の骨組みが取り囲む。}

  \item \ru{Гиганты глядят свысока на рабочих в отсыревшей одежде, которые поднимаются из-под земли, где вот-вот – ожидание затянулось на годы – откроется новая станция, последняя на ветке.}
  
        \en{The giants look down on workers in damp clothes emerging from underground, where any day now—though the waiting has dragged on for years—a new station will open, the last on the line.}
  
        \zh{这些“巨人”俯视着一身潮湿的工人们:他们从地下爬上来,那里的新地铁站眼看就要——等待却拖了好几年——开通,而且还是这条支线的终点站。}
  
        \ja{巨人たちは、湿った作業着の労働者を見下ろしている。彼らは地中から上がってくる。そこでは今にも――待ち時間は何年にも伸びたが――新しい駅が、支線の終点として開業するはずだ。}

  \item \ru{Несколько лет мальчик ходил к площади по грязной тропинке, и ему казалось, что он перемещается в предместье Аида: перед ним возникал строй теней – пенсионеры и несуны с завода продавали метизы, подшипники и еще что-то из подвергшихся насилию металлов; когда метро наконец ожило, тени исчезли.}
  
        \en{For several years the boy walked to the square along a muddy path, and it felt like he was moving into the outskirts of Hades: a lineup of shades appeared—pensioners and factory pilferers selling fasteners, bearings, and other abused bits of metal; when the metro finally came to life, the shades vanished.}
  
        \zh{有好几年,男孩沿着一条泥泞的小路走到广场,他觉得自己像是在通往冥界的郊外:眼前会出现一排“影子”——退休老人和从工厂顺手牵羊的人在卖螺丝五金、轴承,以及别的被“折腾”过的金属零件;地铁终于动起来后,这些影子就消失了。}
  
        \ja{数年のあいだ少年は泥だらけの小道を通って広場へ行き、自分がまるでハデスの郊外へ移動しているように感じていた。目の前に影の列が現れる――年金生活者や工場の「持ち出し屋」たちが、金具やベアリング、その他手荒く扱われた金属類を売っていたのだ。地下鉄がようやく息を吹き返すと、影は消えた。}

  \item \ru{Пока же метро не открылось, дорога мальчика к ближайшей станции и оттуда до школы лежит через подболоченное поле, засеянное бетонными столбами и их обломками.}
  
        \en{But until the metro opens, the boy’s route to the nearest station—and from there to school—runs across a half-swamped field strewn with concrete poles and their broken pieces.}
  
        \zh{可在地铁开通之前,男孩去最近车站、再从那里去学校的路,要穿过一片半沼泽的田地,里面散落着混凝土电线杆和断裂的残块。}
  
        \ja{地下鉄が開通するまでは、少年が最寄りの駅へ、そこから学校へ向かう道は、半ば湿地化した野原を横切る。そこにはコンクリート柱とその破片が点々と立っている。}

  \item \ru{Наверху в проводах гудит электричество.}
  
        \en{Overhead, electricity hums in the wires.}
  
        \zh{头顶的电线里嗡嗡作响。}
  
        \ja{頭上の電線では電気が唸っている。}

  \item \ru{Сереют трубы, а зиккурат фабрики, выпускающей фотоаппараты, сбегает вниз ступеньками – тупыми серыми блоками.}
  
        \en{Pipes gray in the distance, and the ziggurat of a camera factory steps down like a staircase—dull gray blocks.}
  
        \zh{远处灰灰的烟囱立着,一家生产相机的工厂像金字形神塔一样层层向下——一阶阶沉闷的灰色方块。}
  
        \ja{灰色の煙突が並び、カメラ工場のジッグラトが階段状に下っていく――鈍い灰色のブロックとして。}

  \item \ru{Оглянувшись, мальчик форсирует торфяную дорогу и упирается в поле.}
  
        \en{Glancing back, the boy crosses the peat road and comes up against an open field.}
  
        \zh{男孩回头看了一眼,跨过那条泥炭路,便来到一片空旷的田野前。}
  
        \ja{振り返って、少年は泥炭の道を渡りきり、野原に突き当たる。}

  \item \ru{Перед ним заброшенный аэродром и остатки военной базы.}
  
        \en{In front of him are an abandoned airfield and the remains of a military base.}
  
        \zh{他面前是废弃的机场和军事基地的残迹。}
  
        \ja{目の前には放棄された飛行場と軍基地の残骸。}

  \item \ru{Бомбоубежища, накрытые искусственными холмами, и закрытые на замок корпуса, бункеры.}
  
        \en{Bomb shelters covered with artificial mounds, and padlocked buildings—bunkers.}
  
        \zh{被人造土丘覆盖的防空洞,以及上了锁的营房、地堡。}
  
        \ja{人工の丘で覆われた防空壕、南京錠で閉ざされた棟――バンカー。}

  \item \ru{Он взбирается на холм и разглядывает дымящиеся горы мусора вдали.}
  
        \en{He climbs a mound and studies the smoking mountains of trash in the distance.}
  
        \zh{他爬上土丘,望着远处冒烟的垃圾山。}
  
        \ja{少年は丘に登り、遠くに煙を上げるゴミの山を眺める。}

  \item \ru{Когда ветер дует неудачно, сталкер ощущает присутствие ужаса квартирных маклеров – помойки.}
  
        \en{When the wind turns the wrong way, the stalker senses the horror of apartment brokers—the dump.}
  
        \zh{风向不妙时,这个“闯入者”会感受到房产中介们的恐怖存在——垃圾场的气味。}
  
        \ja{風向きが悪いと、この“ストーカー”は不動産ブローカーたちの恐怖――ゴミ捨て場――の存在を感じ取る。}

  \item \ru{Пейзаж как бы говорит: больше трэша, больше ада.}
  
        \en{The landscape seems to say: more trash, more hell.}
  
        \zh{这片风景仿佛在说:更烂一点,更地狱一点。}
  
        \ja{風景が言っているかのようだ――もっとトラッシュを、もっと地獄を。}

  \item \ru{Но в этот раз дым не долетает, пахнет морем.}
  
        \en{But this time the smoke doesn’t reach him; it smells of the sea.}
  
        \zh{但这一次烟味飘不过来,闻到的是海的气息。}
  
        \ja{だが今回は煙が届かず、海の匂いがする。}

  \item \ru{Мальчик, устраиваясь на крыше бункера, открывает книгу и погружается в нее.}
  
        \en{Settling on the roof of a bunker, the boy opens his book and disappears into it.}
  
        \zh{男孩在地堡顶上安顿下来,打开书,沉浸其中。}
  
        \ja{バンカーの屋根に腰を落とすと、少年は本を開き、その中へ沈み込む。}

  \item \ru{Гопники сюда не забредают, да и вообще кругом живет мало людей.}
  
        \en{Gopniks don’t wander out here; in fact, hardly anyone lives around here at all.}
  
        \zh{混混不会跑到这里来,周围本来就没什么人住。}
  
        \ja{ゴプニクがここまで迷い込むことはないし、そもそも周りに人がほとんど住んでいない。}

  \item \ru{Он худой, невысокий, не различает буквы на третьей строчке снизу; логопед не плакал по нему, а пожалуй, справлял тризну.}
  
        \en{He’s skinny, short, can’t make out the letters on the third line from the bottom; a speech therapist wouldn’t have cried over him—more likely would have held a wake.}
  
        \zh{他瘦瘦的、个子不高,看不清下面倒数第三行的字;对他来说,语言治疗师恐怕不会“为他哭泣”,而是会给他办一场葬礼似的哀悼。}
  
        \ja{少年は痩せて背が低く、下から三行目の文字が判別できない。言語療法士は彼のために泣くどころか、むしろ弔いの席を設けたかもしれない。}

  \item \ru{И он торчит часами на крыше в одиночестве, если не считать Сервантеса.}
  
        \en{And he hangs out for hours alone on that roof, if you don’t count Cervantes.}
  
        \zh{于是他一个人一坐就是好几个小时——如果不把塞万提斯算作“陪伴”的话。}
  
        \ja{そして彼は何時間もその屋根に一人で居座る――セルバンテスを数に入れなければ。}

  \item \ru{Он мог бы пригласить с собой братьев, но старший, сводный, Михаил – взрослый, жил отдельно от матери, а средний, Николай, был настолько умен, что не интересовался аэродромом.}
  
        \en{He could have invited his brothers along, but the eldest, a half-brother, Mikhail, was grown and lived apart from their mother, and the middle one, Nikolai, was so smart he had no interest in the airfield.}
  
        \zh{他本可以叫上兄弟们一起,但大哥、同父异母的米哈伊尔已经是成年人,和母亲分开住;二哥尼古拉又聪明到根本对机场这种地方提不起兴趣。}
  
        \ja{兄弟を誘うこともできたが、年長の異父兄ミハイルは大人で母親とは別に暮らしていたし、真ん中のニコライはあまりに賢くて飛行場などに興味がなかった。}

  \item \ru{Когда Николаю было три года, родители сажали его на свободное место в троллейбусе и выдавали книгу – например, «Популярную астрономию».}
  
        \en{When Nikolai was three, their parents would seat him in an empty trolleybus seat and hand him a book—say, Popular Astronomy.}
  
        \zh{尼古拉三岁时,父母会把他放在无轨电车的空座上,递给他一本书——比如《通俗天文学》。}
  
        \ja{ニコライが三歳の頃、両親はトロリーバスの空席に彼を座らせ、本を渡した――たとえば『ポピュラー天文学』のような。}

  \item \ru{Никто из попутчиков не верил, что ребенок ее читает, – рассматривает, поди, – но Николай читал.}
  
        \en{None of the fellow passengers believed the child was actually reading it—just looking at pictures, probably—but Nikolai read.}
  
        \zh{同车的人谁都不信这孩子真在读——大概只是翻着看看吧——可尼古拉就是在读。}
  
        \ja{同乗客の誰もが、その子が本を読んでいるとは信じなかった――せいぜい眺めているだけだろう、と。だがニコライは読んでいた。}

  \item \ru{Он}
  
        \en{[Continues on next page]}
  
        \zh{(下页续)}
  
        \ja{(次ページへ続く)}
\end{itemize}

\end{document}

